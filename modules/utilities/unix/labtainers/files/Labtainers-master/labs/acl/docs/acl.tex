\documentclass[11pt]{article}

\usepackage{times}
\usepackage{epsf}
\usepackage{epsfig}
\usepackage{amsmath, alltt, amssymb, xspace}
\usepackage{wrapfig}
\usepackage{fancyhdr}
\usepackage{url}
\usepackage{verbatim}
\usepackage{fancyvrb}
\usepackage{float}

\usepackage{subfigure}
\usepackage{cite}
\usepackage{hyperref}
\hypersetup{%
    pdfborder = {0 0 0}
}
\topmargin      -0.50in  % distance to headers
\oddsidemargin  0.0in
\evensidemargin 0.0in
\textwidth      6.5in
\textheight     8.9in 


%\centerfigcaptionstrue

%\def\baselinestretch{0.95}


\newcommand\discuss[1]{\{\textbf{Discuss:} \textit{#1}\}}
%\newcommand\todo[1]{\vspace{0.1in}\{\textbf{Todo:} \textit{#1}\}\vspace{0.1in}}
\newtheorem{problem}{Problem}[section]
%\newtheorem{theorem}{Theorem}
%\newtheorem{fact}{Fact}
\newtheorem{define}{Definition}[section]
%\newtheorem{analysis}{Analysis}
\newcommand\vspacenoindent{\vspace{0.1in} \noindent}

%\newenvironment{proof}{\noindent {\bf Proof}.}{\hspace*{\fill}~\mbox{\rule[0pt]{1.3ex}{1.3ex}}}
%\newcommand\todo[1]{\vspace{0.1in}\{\textbf{Todo:} \textit{#1}\}\vspace{0.1in}}

%\newcommand\reducespace{\vspace{-0.1in}}
% reduce the space between lines
%\def\baselinestretch{0.95}

\newcommand{\fixmefn}[1]{ \footnote{\sf\ \ \fbox{FIXME} #1} }
\newcommand{\todo}[1]{
\vspace{0.1in}
\fbox{\parbox{6in}{TODO: #1}}
\vspace{0.1in}
}

\newcommand{\mybox}[1]{
\vspace{0.2in}
\noindent
\fbox{\parbox{6.5in}{#1}}
\vspace{0.1in}
}


\newcounter{question}
\setcounter{question}{1}

\newcommand{\myquestion} {{\vspace{0.1in} \noindent \bf Question \arabic{question}:} \addtocounter{question}{1} \,}

\newcommand{\myproblem} {{\noindent \bf Problem \arabic{question}:} \addtocounter{question}{1} \,}



\newcommand{\copyrightnotice}[1]{
\vspace{0.1in}
\fbox{\parbox{6in}{
      This lab was developed for the Labtainer framework by the Naval Postgraduate 
      School, Center for Cybersecurity and Cyber Operations.
      This work is in the public domain, and cannot be copyrighted.}}
\vspace{0.1in}
}


\newcommand{\idea}[1]{
\vspace{0.1in}
{\sf IDEA:\ \ \fbox{\parbox{5in}{#1}}}
\vspace{0.1in}
}

\newcommand{\questionblock}[1]{
\vspace{0.1in}
\fbox{\parbox{6in}{#1}}
\vspace{0.1in}
}


\newcommand{\argmax}[1]{
\begin{minipage}[t]{1.25cm}\parskip-1ex\begin{center}
argmax
#1
\end{center}\end{minipage}
\;
}

\newcommand{\bm}{\boldmath}
\newcommand  {\bx}    {\mbox{\boldmath $x$}}
\newcommand  {\by}    {\mbox{\boldmath $y$}}
\newcommand  {\br}    {\mbox{\boldmath $r$}}


\newcommand{\tstamp}{\today}   
%\rfoot[\fancyplain{\tstamp} {\tstamp}]  {\fancyplain{}{}}

\pagestyle{fancy}
\lhead{\bfseries Labtainers}
\chead{}
\rhead{\small \thepage}
\lfoot{}
\cfoot{}
\rfoot{}




\begin{document}

\begin{center}
{\LARGE Linux Access Control Lists (ACLs)}
\vspace{0.1in}\\
\end{center}

\copyrightnotice

\section{Overview}
This exercise explores the use of Linux ACLs to provide
access control over files, with more flexibility than 
the access control offered by traditional UNIX file permissions.
It is assumed the student has received instruction, or independent
study, in access control policies and ACLs.  A description of
Linux ACLs can be found at 
\url {https://wiki.archlinux.org/index.php/Access\_Control\_Lists}

\section{Lab Environmnet}
This lab runs in the Labtainer framework,
available at http://my.nps.edu/web/c3o/labtainers.
That site includes links to a pre-built virtual machine
that has Labtainers installed, however Labtainers can
be run on any Linux host that supports Docker containers.

From your labtainer-student directory start the lab using:
\begin{verbatim}
    labtainer acl
\end{verbatim}
Links to this lab manual will be displayed.  

\section{Setup}
After starting the lab, three virtual terminals will be created, each with a login
prompt.  Login to these as three different users:

\medskip
{\begin{tabular}{ l | l  }

 \hline
 user & password \\ \hline
 bob & password4bob \\
 alice & password4alice \\
 harry & password4harry \\
\end{tabular}

\section{Lab Tasks}
In this lab, you will use the {\tt getfacl} and {\tt setfacl} commands to 
view and modify ACLs on files. Use the -h option to learn about these
commands, e.g., {\tt getfacl -h}.

\subsection{Review existing file permissions}
In the "alice" terminal, cd to the /shared\_data directory and list
the files:
\begin{verbatim}
   cd /shared_data
   ls -l
\end{verbatim}
\noindent
Observe the file permissions on the {\tt accounting.txt} file and the two
directories.  Would you expect Alice to be able to view the content of
{\tt accounting.txt}?  Try to cat the file.

Look again at the directory listing.  Note how the {\tt accounting.txt}
entry has a permission setting of:
\begin{verbatim}
    -rw-rw----+
\end{verbatim}
That {\tt +} symbol at the end indicates that this file has an ACL in addition
to the standard UNIX permissions of "rw" for user and group.  You can view the
ACL permissions on this file using:
\begin{verbatim}
   getfacl accounting.txt
\end{verbatim}

Note how one of the three users has permission to modify that file?  Go to that
user's virtual terminal and append to the file using:
\begin{verbatim}
   echo "more stuff" >> /shared_data/accounting.txt
\end{verbatim}

Back in the Alice terminal, try to modify that file to confirm she lacks that access.


\subsection{Task 2: Set an ACL on a single file}
As the Bob user, use the {\tt setfacl} command to allow Alice to read the
{\tt /shared\_data/bob/bobstuff.txt file}.  Then, as the Alice user,
confirm her ability to read the file.  And as the Harry user, confirm
he lacks permission to read the file.


\subsection{Task 2: Set an default ACL for a directory}
As the Alice user, we want to define a default ACL such that, whenever
Alice creates a new file in the {\tt /shared\_data/alice} directory,
that new file will be readable by Bob, but not by users other than
Bob and Alice.  Consider doing this systematically:
\begin{itemize}
\item Create a file in {\tt /shared\_data/alice} and check its permissions
\item Set the default ACL on the alice directory to permit bob to read newly created files. 
\item Create another new file in {\tt /shared\_data/alice} and check its permissions.  Are they what you expect?
\item Revise your default ACL on the alice directory if necessary
\item Confirm permissions on a newly created file are as desired.
\end{itemize}

\subsection{Task 3: Trojan Horses}
Revisit the permissons on the {\tt /shared\_data/accounting.txt} file.  Bob is
unable to read this file, but he would very much like to know its content.
Bob knows Alice is a fool for ascii art, and he created the {\tt /shared\_data/bob/fun}
script.  As Bob, modify that script so that if Alice (or Harry) run the script, it will
make a copy of the accounting.txt file in a manner that allows Bob to see the content.
Confirm that when Bob runs this script, it does not provide him access to the data.  But
when it is run by Alice, then Bob gets access to the information.  Note the distinction beween
Bob getting access to the file, and Bob getting access to the information.

\section{Submission}
After finishing the lab, go to the terminal on your Linux system that was used to start the lab and type:
\begin{verbatim}
    stoplab acl
\end{verbatim}
When you stop the lab, the system will display a path to the zipped lab results on your Linux system.  Provide that file to 
your instructor, e.g., via the Sakai site.

\end{document}
