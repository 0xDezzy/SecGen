\documentclass[11pt]{article}

\usepackage{times}
\usepackage{epsf}
\usepackage{epsfig}
\usepackage{amsmath, alltt, amssymb, xspace}
\usepackage{wrapfig}
\usepackage{fancyhdr}
\usepackage{url}
\usepackage{verbatim}
\usepackage{fancyvrb}
\usepackage{float}

\usepackage{subfigure}
\usepackage{cite}
\usepackage{hyperref}
\hypersetup{%
    pdfborder = {0 0 0}
}
\topmargin      -0.50in  % distance to headers
\oddsidemargin  0.0in
\evensidemargin 0.0in
\textwidth      6.5in
\textheight     8.9in 


%\centerfigcaptionstrue

%\def\baselinestretch{0.95}


\newcommand\discuss[1]{\{\textbf{Discuss:} \textit{#1}\}}
%\newcommand\todo[1]{\vspace{0.1in}\{\textbf{Todo:} \textit{#1}\}\vspace{0.1in}}
\newtheorem{problem}{Problem}[section]
%\newtheorem{theorem}{Theorem}
%\newtheorem{fact}{Fact}
\newtheorem{define}{Definition}[section]
%\newtheorem{analysis}{Analysis}
\newcommand\vspacenoindent{\vspace{0.1in} \noindent}

%\newenvironment{proof}{\noindent {\bf Proof}.}{\hspace*{\fill}~\mbox{\rule[0pt]{1.3ex}{1.3ex}}}
%\newcommand\todo[1]{\vspace{0.1in}\{\textbf{Todo:} \textit{#1}\}\vspace{0.1in}}

%\newcommand\reducespace{\vspace{-0.1in}}
% reduce the space between lines
%\def\baselinestretch{0.95}

\newcommand{\fixmefn}[1]{ \footnote{\sf\ \ \fbox{FIXME} #1} }
\newcommand{\todo}[1]{
\vspace{0.1in}
\fbox{\parbox{6in}{TODO: #1}}
\vspace{0.1in}
}

\newcommand{\mybox}[1]{
\vspace{0.2in}
\noindent
\fbox{\parbox{6.5in}{#1}}
\vspace{0.1in}
}


\newcounter{question}
\setcounter{question}{1}

\newcommand{\myquestion} {{\vspace{0.1in} \noindent \bf Question \arabic{question}:} \addtocounter{question}{1} \,}

\newcommand{\myproblem} {{\noindent \bf Problem \arabic{question}:} \addtocounter{question}{1} \,}



\newcommand{\copyrightnotice}[1]{
\vspace{0.1in}
\fbox{\parbox{6in}{
      This lab was developed for the Labtainer framework by the Naval Postgraduate 
      School, Center for Cybersecurity and Cyber Operations.
      This work is in the public domain, and cannot be copyrighted.}}
\vspace{0.1in}
}


\newcommand{\idea}[1]{
\vspace{0.1in}
{\sf IDEA:\ \ \fbox{\parbox{5in}{#1}}}
\vspace{0.1in}
}

\newcommand{\questionblock}[1]{
\vspace{0.1in}
\fbox{\parbox{6in}{#1}}
\vspace{0.1in}
}


\newcommand{\argmax}[1]{
\begin{minipage}[t]{1.25cm}\parskip-1ex\begin{center}
argmax
#1
\end{center}\end{minipage}
\;
}

\newcommand{\bm}{\boldmath}
\newcommand  {\bx}    {\mbox{\boldmath $x$}}
\newcommand  {\by}    {\mbox{\boldmath $y$}}
\newcommand  {\br}    {\mbox{\boldmath $r$}}


\newcommand{\tstamp}{\today}   
%\rfoot[\fancyplain{\tstamp} {\tstamp}]  {\fancyplain{}{}}

\pagestyle{fancy}
\lhead{\bfseries Labtainers}
\chead{}
\rhead{\small \thepage}
\lfoot{}
\cfoot{}
\rfoot{}





%\documentclass{article} 
%\usepackage{graphicx}
%\usepackage{color}
%\usepackage[latin1]{inputenc}
%\usepackage{lgrind}

%\input {highlight.sty}

\lhead{\bfseries SEED Labs -- Format String Vulnerability Lab}


\begin{document}

\begin{center}
{\LARGE Format String Vulnerability Lab}
\end{center}

\copyrightnotice


\section{Lab Overview}

The learning objective of this lab is for students to gain the first-hand
experience on format-string vulnerability by putting what they have learned 
about the vulnerability from class into actions.
The format-string vulnerability is caused by code like 
{\tt printf(user\_input)}, where the contents of variable
of {\tt user\_input} is provided by users. 
This {\tt printf}
statement can become dangerous, because it can lead to one of the following
consequences: (1) crash the 
program, (2) read from an arbitrary memory place, and (3) modify
the values of an arbitrary memory location. The last consequence
is very dangerous because it can allow users to modify internal
variables of the program, and thus change the behavior
of the program.  


In this lab, students will be given a program with a format-string
vulnerability; their task is to develop a scheme to exploit
the vulnerability.  In addition to the
attacks, students will be guided to walk through a protection
scheme that can be used to defeat this type of attacks. 
Students need to evaluate
whether the scheme work or not and explain why.


\subsection{Getting Started}
The lab is started from the Labtainer working 
directory on your Docker-enabled host, e.g., a Linux VM.
From there, issue the command:
\begin{verbatim}
    labtainer formatstring
\end{verbatim}

The resulting virtual terminal will include 
a bash shell.  The vuln\_prog.c 
described below be in your home directory.

\section{Lab Tasks}

\subsection{Task 1: Exploit the vulnerability}
Before beginning this task, nsure that Address Space Layout Randomization (ASLR) is enabled:
\begin{verbatim}
      sudo sysctl -w kernel.randomize_va_space=2
\end{verbatim}
In the following program, you will be asked to provide an input, which
will be saved in a buffer called {\tt user\_input}. 
The program then prints out the buffer using {\tt printf}.
Unfortunately, there is a format-string vulnerability in the way
the {\tt printf} is called on the user inputs. We want to exploit
this vulnerability and see what we can achieve.

The program has two secret values stored in its memory,
and you are interested in these secret values. However, 
the secret values are unknown to you, nor can you find them from reading the 
binary code (for the sake of simplicity, we hardcode the secrets 
using constants 0x$44$ and 0x$55$). 
Although you do not know the secret values, in practice, 
it is not so difficult to find out 
the memory address (the range or the exact value) of them (they are 
in consecutive addresses), because
for many operating systems, the addresses are exactly the same anytime you 
run the program. 
In this lab, we just assume that you have already known the exact addresses. 
To achieve this,
the program ``intentionally'' prints out the addresses for you. With such
knowledge, your goal is to achieve the followings (not necessarily at the 
same time): 
\begin{itemize} 
\item Crash the program.
\item Print out the secret[1] value.
\item Modify the secret[1] value.
\item Modify the secret[1] value to equal 0xa.
\end{itemize} 

For this lab task, you are not to modify the code. Namely, you need to achieve
the above objectives without modifying the vulnerable code.
However, you do have a copy of the source code, which can help 
you design your attacks.
 
\begin{lstlisting}
/* vul_prog.c */
#include<stdio.h>
#include<stdlib.h>

#define SECRET1 0x44
#define SECRET2 0x55

int main(int argc, char *argv[])
{
    char user_input[100];
    int *secret;
    int int_input;
    int a, b, c, d; /* other variables, not used here.*/
    
    /* The secret value is stored on the heap */
    secret = (int *) malloc(2*sizeof(int));
    
    /* getting the secret */
    secret[0] = SECRET1; secret[1] = SECRET2;
    
    printf("The variable secret's address is 0x%8x (on stack)\n", 
                        (unsigned int)&secret);
    printf("The variable secret's value is 0x%8x (on heap)\n", 
                        (unsigned int)secret);
    printf("secret[0]'s address is 0x%8x (on heap)\n", 
                        (unsigned int)&secret[0]);
    printf("secret[1]'s address is 0x%8x (on heap)\n", 
                        (unsigned int)&secret[1]);
    
    printf("Please enter a decimal integer\n");
    scanf("%d", &int_input);  /* getting an input from user */
    printf("Please enter a string\n");
    scanf("%s", user_input); /* getting a string from user */
    
    /* Vulnerable place */
    printf(user_input);
    printf("\n");
    
    /* Verify whether your attack is successful */
    printf("The original secrets: 0x%x -- 0x%x\n", SECRET1, SECRET2);
    printf("The new secrets:      0x%x -- 0x%x\n", secret[0], secret[1]);
    return 0;
}
\end{lstlisting}


\paragraph{Hints:}
From the printout, you will find out that {\tt secret[0]} and 
{\tt secret[1]} are located on the heap, i.e., the actual
secrets are stored on the heap. We also know that 
the address of the first secret (i.e., the value of 
the variable {\tt secret}) can be found on the stack, because
the variable {\tt secret} is allocated on the stack. 
In other words, if you want to overwrite {\tt secret[0]}, its address 
is already on the stack; your format string can  take advantage 
of this information. However, although {\tt secret[1]} is just
right after {\tt secret[0]}, its address is not available on the 
stack. This poses a major challenge for your format-string exploit, 
which needs to have the exact address right on the stack in order 
to read or write to that address.
 


\subsection{Task 2: Memory randomization}

If the first scanf statement ({\tt scanf("\%d", int\_input)})
does not exist, i.e., the program does not ask you to
enter an integer, the attack in Task 1 become more difficult 
for those operating systems that have implemented address 
randomization. Pay attention to the address of secret[0] (or secret[1]). 
When you run the program once again, will you get the same
address?

Address randomization is introduced to make a number of 
attacks difficult, such as buffer overflow, format string, etc. 
To appreciate the idea of address randomization, we will turn
off the address randomization in this task, and see whether the 
format string attack on the previous vulnerable program (without
the first scanf statement) is still difficult. You can use the following
command to turn off the address randomization:

\begin{verbatim}
   sudo sysctl -w kernel.randomize_va_space=0
\end{verbatim}


After turning off the address randomization, your task is to 
repeat the same task described in Task 1, but you have to 
remove the first scanf statement ({\tt scanf("\%d", int\_input)})
from the vulnerable program, and recompile it using:

\begin{verbatim}
    gcc -m32 -fno-stack-protector -g -o vul_prog vul_prog.c
\end{verbatim}

\paragraph{How to let {\tt scanf} accept an arbitrary number?}
Usually, {\tt scanf} is going to pause for you to type inputs. 
Sometimes, you want the program to take a number 0x05 (not the 
character `5'). Unfortunately, when you type `5' at the input, {\tt scanf}
actually takes in the ASCII value of `5', which is 0x$35$, rather
than 0x$05$. The challenge is that in ASCII, 0x$05$ is not a typable
character, so there is no way we can type in this value. 
One way to solve this problem is to use a file. 
We can easily write a C program that stores 0x05 (again, not `5')  
to a file (let us call it {\tt mystring}), then we can run the vulnerable program (let us 
call it {\tt a.out}) with its input being redirected to {\tt mystring}; 
namely, we run {\tt "a.out < mystring"}. This way, {\tt scanf} will take its 
input from the file {\tt mystring}, instead of from the keyboard.

You need to pay attention to some special numbers, such as 0x$0A$ (newline), 
0x$0C$ (form feed), 0x$0D$ (return), and 0x$20$ (space). 
{\tt scanf} considers them as separator, and will stop reading anything
after these special characters if we have only one {\tt "\%s"} in
{\tt scanf}.  If one of these special numbers are in the address, 
you have to find ways to get around this.
To simplify your task, if you are unlucky and the secret's address 
happen to have those special numbers in it, we allow you to add another {\tt malloc}
statement before you allocate memory for {\tt secret[2]}. This extra
{\tt malloc} can cause the address of secret values to change. If you
give the {\tt malloc} an appropriate value, you can 
create a ``lucky'' situation, where the addresses of secret do not 
contain those special numbers. 

The following program writes a format string into a file called {\tt mystring}.
The first four bytes consist of an arbitrary number that you want 
to put in this format string, followed by the rest of format string 
that you typed in from your keyboard.

\begin{lstlisting}
/* write_string.c */

#include <sys/types.h>
#include <stdio.h>
#include <string.h>
#include <sys/stat.h>
#include <fcntl.h>

int main()
{
    char buf[1000];
    int fp, size;
    unsigned int *address;
    
    /* Putting any number you like at the beginning of the format string */
    address = (unsigned int *) buf;
    *address = 0x804b01c;
    
    /* Getting the rest of the format string */
    scanf("%s", buf+4);
    size = strlen(buf+4) + 4;
    printf("The string length is %d\n", size);
    
    /* Writing buf to "mystring" */
    fp = open("mystring", O_RDWR | O_CREAT | O_TRUNC, S_IRUSR | S_IWUSR);
    if (fp != -1) {
        write(fp, buf, size);
        close(fp);
    } else {
        printf("Open failed!\n");
    }
}
\end{lstlisting}


\section{Guidelines} 


\subsection{What is a format string?}

\begin{verbatim}
             printf ("The magic number is: %d\n", 1911);
\end{verbatim}

The text to be printed is ``The magic number is:'', followed by a format
parameter `\%d', which is replaced with the parameter (1911) in the output.
Therefore the output looks like: The magic number is: 1911. In addition to
\%d,
there are several other format parameters, each having different meaning.
The
following table summarizes these format parameters:
\begin{Verbatim}[frame=single]
Parameter       Meaning                               Passed as
-------------------------------------------------------------------
   %d        decimal (int)                              value
   %u        unsigned decimal (unsigned int)            value
   %x        hexadecimal (unsigned int)                 value
   %s        string ((const) (unsigned) char *)         reference
   %n        number of bytes written so far, (* int)    reference
\end{Verbatim}

\subsection{The Stack and Format Strings}


The behavior of the format function is controlled by the format string.
The function retrieves the parameters requested by the format string
from the stack.

\begin{Verbatim}[frame=single]
printf ("a has value %d, b has value %d, c is at address: %08x\n", 
                          a, b, &c);
\end{Verbatim}


\begin{center}
\includegraphics*[viewport=1.10in 5.70in 6.1in 9.5in,width=3.5in]{Figs/Format_String_Stack.pdf}
\end{center}


\subsection{What if there is a miss-match}

What if there is a miss-match between the format string and the actual
      arguments?
\begin{Verbatim}[frame=single]
printf ("a has value %d, b has value %d, c is at address: %08x\n", 
                          a, b);
\end{Verbatim}

   \begin{itemize}
   \item In the above example, the format string asks for 3 arguments, but the
         program actually provides only two (i.e. $a$ and $b$).
   \item Can this program pass the compiler?
       \begin{itemize}
       \item The function {\tt printf()} is defined as function with variable
             length of arguments. Therefore, by looking at the number of
             arguments, everything looks fine.
       \item To find the miss-match, compilers needs to understand how
             {\tt printf()} works and what the meaning of the format string
             is. However, compilers usually do not do this kind of analysis.
       \item Sometimes, the format string is not a constant string, it is
             generated during the execution of the program. Therefore, there
             is no way for the compiler to find the miss-match in this case.
       \end{itemize}
   \item Can {\tt printf()} detect the miss-match?
       \begin{itemize}
       \item The function {\tt printf()} fetches the arguments from the stack.
             If the format string needs 3 arguments, it will fetch 3 data items
             from the stack.
             Unless the stack is marked with a boundary, {\tt printf()} does not
             know that it runs out of the arguments that are provided to it.
       \item Since there is no such a marking. {\tt printf()} will continue
             fetching data from the stack. In a miss-match case, it will
             fetch some data that do not belong to this function call.
       \end{itemize}

   \item What trouble can be caused by {\tt printf()} when it starts to
         fetch data that is meant for it?

   \end{itemize}



\subsection{Viewing Memory at Any Location}


   \begin{itemize}
   \item We have to supply an address of the memory. However, we cannot change
   the code; we can only supply the format string.
   \item If we use {\tt printf(\%s)} without specifying a memory address, the
   target address will be obtained from the stack anyway by the {\tt printf()}
   function. The function maintains an initial stack pointer, so it knows the location
   of the parameters in the stack.

   \item Observation: the format string is usually located on the stack. If we
   can encode the target address in the format string, the target address will
   be in the stack. In the following example, the format string is stored in
   a buffer, which is located on the stack.

\begin{lstlisting}
int main(int argc, char *argv[])
{
  char user_input[100];
  ... ... /* other variable definitions and statements */

  scanf("%s", user_input); /* getting a string from user */
  printf(user_input); /* Vulnerable place */

  return 0;
}
\end{lstlisting}


   \item If we can force the printf to obtain the address from the format
   string (also on the stack), we can control the address.

\begin{verbatim}
printf ("\x10\x01\x48\x08  %x %x %x %x %s");
\end{verbatim}


   \item \verb|\x10\x01\x48\x08| are the four bytes of the target address.
         In C language, \verb|\x10| in a string tells the compiler to put 
         a hexadecimal value \verb|0x10| in the current position. The value
         will take up just one byte. Without using \verb|\x|, if we directly
         put \verb|"10"| in a string, the ASCII values of the characters \verb|'1'|
         and \verb|'0'| will be stored. Their ASCII values are
         $49$ and $48$, respectively.

   \item {\tt \%x}  causes the stack pointer to move towards the format string.

   \item Here is how the attack works if {\tt user\_input[]} contains the
         following format string:

\begin{verbatim}      
"\x10\x01\x48\x08  %x %x %x %x %s".
\end{verbatim}
\begin{center}
\includegraphics*[viewport=0.6in 4.10in 6.5in 8.1in,width=5.5in]{Figs/Format_String_Attack.pdf}
\end{center}

   \item Basically, we use four \verb|%x| to move the {\tt printf()}'s pointer
         towards the address that we stored in the format string. Once we reach
         the destination, we will give \verb|%s| to {\tt print()}, causing it
         to print out the contents in the memory address \verb|0x10014808|.
         The function {\tt printf()} will treat the contents as a string, and
         print out the string until reaching the end of the string (i.e.  0).

   \item The stack space between {\tt user\_input[]} and the address passed to
         the {\tt printf()} function is not for {\tt printf()}. However, because
         of the format-string vulnerability in the program, {\tt printf()}
         considers them as the arguments to match with the \verb|%x| in
         the format string.

   \item The key challenge in this attack is to figure out the distance between the
         {\tt user\_input[]} and the address passed to the {\tt printf()} function.
         This distance decides how many \verb|%x| you need to insert into the
         format string, before giving \verb|%s|.


   \end{itemize}




\subsection{Writing an Integer to Memory}

   \begin{itemize}
   \item {\tt \%n}: The number of characters written so far is stored into the
   integer indicated by the corresponding argument.
   \begin{Verbatim}[frame=single]
      int i;
      printf ("12345%n", &i);
   \end{Verbatim}
   \item It causes {\tt printf()} to write 5 into variable $i$.

   \item Using the same approach as that for viewing memory at any location, we
   can cause {\tt printf()} to write an integer into any location. Just replace
   the \verb|%s| in the above example with \verb|%n|, and the contents at the
   address \verb|0x10014808| will be overwritten.

   \item Using this attack, attackers can do the following:
       \begin{itemize}
       \item Overwrite important program flags that control access privileges
       \item Overwrite return addresses on the stack, function pointers, etc.
       \end{itemize}

   \item However, the value written is determined by the number of characters
   printed before the {\tt \%n} is reached. Is it really possible to write
   arbitrary integer values?
       \begin{itemize}
       \item Use dummy output characters. To write a value of 1000, a simple
       padding of 1000 dummy characters would do.
       \item To avoid long format strings, we can use a width specification of
       the format indicators.
       \end{itemize}
   \end{itemize}







\section{Submission}
When the lab is completed, or you'd like to stop working for a while, run
\begin{verbatim}
    stoplab formatstring
\end{verbatim}

from the host Labtainer working directory.  You can always restart the
Labtainer to continue your work.  When the Labtainer is stopped, a
zip file is created and copied to a location displayed by the stoplab
command.  When the lab is completed, send that zip file to the instructor.

You need to submit a detailed lab report to describe what you have
done and what you have observed; you also need to provide explanation
to the observations that are interesting or surprising.

\end{document}
